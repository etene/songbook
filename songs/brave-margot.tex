\beginsong{Brave Margot}[by={Georges Brassens}]


% http://beausoleil.free.fr/tab.php?texte=brave_margot.html&defaut=1&type=1&anglo=1

% dans la version originale les Cm sont des Cdim. mais ça passe mieux au ukulele en Cm.

\beginverse
Margo\[^C]ton la jeune ber\[G7]gère,
Trouvant dans l'herbe un petit \[C]chat
Qui ve\[Am]{nait de} perdre sa mère,
L'adop\[D7]ta. \[G7]

Elle entrouvre sa collerette
Et le couche contre son \[C]sein.
C'était \[Am]{tout c'qu'elle avait}, pau\[Dm]vrette,
\[G7]Comme cous\[C]sin.

Le chat, \[Am]{la prenant} pour sa \[Em]mère,
Se mit \[Am]{à téter} tout de \[Em]go.
Émue \[Am]Margot le laissa \[Em]faire,
Brave Mar\[Am]got !

Un croquant, passant à la \[Em]ronde,
Trouvant \[Am]{le tableau} peu com\[Em]mun,
S'en al\[Am]{la le dire} à tout l'\[Em]monde,
Et, le lende\[D7]main... \[G7]
\endverse

\beginchorus
Quand Margot dégrafait son cor\[C]sage
Pour donner la gou\[Cm]goutte à son \[C]chat,
Tous les gars, tous les \[A7]gars du vil\[D7]la\[G7]ge
Étaient \[C]{là là là là là là là}, étaient \[G#7]{là là là là là là}

Et Mar\[G7]got, qu'était simple et très \[C]sage,
Présumait qu'c'était \[Cm]{pour voir} son \[C]chat
Qu'tous les gars, qu'tous les \[A7]gars du vil\[D7]la\[G7]ge,
Étaient \[C]{là là là là là là là}, étaient \[G#7]{là là là là là là}
\endchorus

\beginverse
L'maître d'é^cole et ses potaches,
Le maire, le bedeau, le bou^gnat,
Négli^geaient carrément leur tâche
Pour voir ^ça. ^

Le facteur, d'ordinaire si preste,
Pour voir ça ne distribuait ^plus
Les let^tres que personne, au ^reste,
^N'aurait ^lues.

Pour voir ^ça - Dieu le leur par^donne ! -
Les en^fants de chœur, au mi^lieu
Du saint ^sacrifice, aban^donnent
Le saint ^lieu.

Les gendarmes, même les gen^darmes,
Qui sont ^{par nature} si bal^lots,
Se lais^saient toucher par les ^charmes
Du joli ta^bleau.^
\endverse

\textnote{Refrain}

\beginverse
Mais les ^autres femmes de la commune,
Privées d'leurs époux, d'leurs ga^lants,
Accumu^lèrent la rancune,
Patiem^ment.^

Puis un jour, ivres de colère,
Elles s'armèrent de bâ^tons
Et, fa^rouches, elles immo^lèrent
^{Le cha}^ton.

La ber^gère, après bien des ^larmes,
Pour s'con^soler prit un ma^ri,
Et ne ^dévoila plus ses ^charmes
Que pour ^lui.

Le temps passa sur les mé^moires,
On ou^blia l'événe^ment,
Seuls les ^vieux racontent en^core
À leurs p'tits en^fants...^
\endverse

\textnote{Refrain}

\beginchorus
Étaient \[C]{là là là là là} \[G#7]{là là là là là} \[C]là\[G7]\[C]
\endchorus

\endsong