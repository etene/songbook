\beginsong{Je suis un homme}[by=Michel Polnareff]

\printG
\printGseven
\printC
\printD
\printDseven
\printAm
\printEm


% TODO: y'a pas juste des D, mais des D7 ou D4

\beginverse
\textnote{Version Simon: remplacer "pédé" par "penne"}
La soc\[G]iété ayant reno\[G7]ncé à me transf\[C]ormer,
À me dé\[G]guiser, pour lui ressemb\[D]ler,
Les gens qui me \[G]voient passer dans la \[G7]rue me traitent de p\[C]édé
Mais les femmes qui le \[G]croient \[D]
N'ont qu'à m'es\[G]sayer \[D]
\endverse

\beginchorus
\memorize[chorus]
Je suis un \[G]homme, je suis un \[G7]homme
Quoi de plus \[C]{naturel} e\[Am]{n somme}
Au lit mon \[G]style correspond \[Em]bien à mon état \[Am]civil \[D]
Je suis un \[G]homme, je suis un \[G7]homme
Comme on en \[C]voit dans les m\[Am]uséums
Un jules, u\[G]{n vrai}, un boute-en\[Em]{-train} toujours prêt, toujour\[Am]{s gai} \[D]
\endchorus

\beginverse
A mon pro^cès, moi j'ai fait ci^{ter une} foule de té^moins
Toutes les filles du ^coin, qui me connaissaient ^bien
Quand le prési^dent m'a interro{^gé, j'ai} prêté ser^ment,
J'ai pris ma plus belle ^voix,^{ et j'ai} décla^ré: ^
\endverse

\beginchorus
\replay[chorus]
Je suis un ^homme, je suis un ^homme
Quoi de plus ^{naturel} e^{n somme}
Au lit mon ^style correspond ^bien à mon état ^civil ^
Je suis un ^homme, je suis un ^homme
Pas besoin ^d'un ^référendum
Ni d'un ex^pert pour consta^ter qu'elles sont en nombre ^pair ^
\endchorus

\beginverse
\textnote{Version Simon: "ce serait une vis de marcher tout nu sur les ammonites"}
En soixante-^dix, il n'est pas ques^tion, ce serait du ^vice
De marcher tout ^nu sur les ave^nues
Mais c'est pour de^main, et un de ce j^ours, quand je chante^rai
Aussi nu qu'un tam^bour,^ vous verrez bien ^que ^
\endverse

\beginchorus
\replay[chorus]
Je suis un ^homme, je suis un ^homme
Et de là-^haut sur mon p^odium
J'ébloui^rai le tout Pa^ris de mon anato^mie ^

Je suis un ^homme, je suis un ^homme
Quoi de plus ^{naturel} e^{n somme}
Au lit mon ^style correspond ^bien à mon état ^civil ^
\endchorus
\textnote{bis}
\endsong