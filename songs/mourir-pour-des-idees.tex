\beginsong{Mourir pour des idées}[by={Georges Brassens}]

% http://beausoleil.free.fr/GB.Mourir.Pour.Des.Idees.php?titre=Mourir+pour+des+id%C3%A9es&anglo=1

\beginverse
\[Am]Mourir pour \[C]des i\[D]dées, \[Am]l'idée est \[C]excel\[D]lente.\[Am]
Moi, j'ai fail\[C]{li mou}\[D]{rir de} \[E7]{ne l'avoir} pas \[Am]eue,

\[Am]{Car tous ceux} \[C]qui l'a\[D]vaient, \[Am]multitude \[C]acca\[D]blante,\[Am]
En hurlant \[C]{à la} \[D]mort me \[E7]sont tombés des\[Am]sus.

\[Dm] Ils ont su me convaincre\[G7]{ et ma muse} insolente,\[C]
Abjurant ses erreurs, se rallie à leur foi\[E7]

Avec un soupçon de réserve toutefois :
Mou\[Am]rons \[C]pour des idées, d'acc\[F]ord, mais de mort \[G7]len-en-\[C]te,
D'ac\[F]cord, mais de mort \[E7]len-en-en\[Am]te.
\endverse

\beginverse
^{Des idées} ^récla^mant ^{le fameux} ^sacri^fice,^
Les sectes ^de tout ^poil en ^offrent des sé^quelles,

^{Et la} question ^{se po}^se ^{aux victi}^mes no^vices:^
Mourir ^pour des i^dées, c'est ^bien beau, mais les^quelles ?

^ Et comme toutes sont^{ entre} elles ressemblantes,^
Quand il les voit venir, avec leur gros drapeau,^

Le sage, en hésitant, tourne autour du tombeau.
Mou^rons ^pour des idées, d'ac^cord, mais de mort ^len-en-^te,
D'ac^cord, mais de mort ^len-en-en^te.
\endverse

\beginverse
^Encor s'il ^suffi^sait ^{de quelques} ^héca^tombes^
Pour qu'enfin ^tout chan^geât, qu'en^{fin tout} s'arran^geât !

^Depuis tant ^{de grands} ^soirs ^{que tant} de ^têtes ^tombent,^
Au para^dis sur ^terre on ^{y serait} dé^jà.

^ Mais l'âge d'or sans cesse^{ est} remis aux calendes,^
Les dieux ont toujours soif, n'en ont jamais assez,^

Et c'est la mort, la mort toujours recommencée.
Mou^rons ^pour des idées, d'ac^cord, mais de mort ^len-en-^te,
D'ac^cord, mais de mort ^len-en-en^te.
\endverse


\beginverse
^{Ô vous}, les ^boute^feux, ^{ô vous} les ^bons a^pôtres,^
Mourez donc ^les pre^miers, nous ^vous cédons le ^pas.

^Mais de grâ^ce, mor^bleu ! ^laissez vi^vre les ^autres !^
La vie est ^{à peu} ^près leur ^seul luxe ici-^bas ;

^ Car enfin, la Camarde^{ est} assez vigilante,^
Elle n'a pas besoin qu'on lui tienne la faux.^

Plus de danse macabre autour des échafauds !
Mou^rons ^pour des idées, d'ac^cord, mais de mort ^len-en-^te,
D'ac^cord, mais de mort ^len-en-en^te. \[E7] \[Am]
\endverse

\endsong